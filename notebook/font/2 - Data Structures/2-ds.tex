\documentclass[11pt, oneside]{article}   	% use "amsart" instead of "article" for AMSLaTeX format
\usepackage{geometry}                		% See geometry.pdf to learn the layout options. There are lots.
\geometry{a4paper}                   		% ... or a4paper or a5paper or ... 
%\geometry{landscape}                		% Activate for for rotated page geometry
\usepackage[parfill]{parskip}    		% Activate to begin paragraphs with an empty line rather than an indent
\usepackage{graphicx}	
\usepackage{fancyhdr}
\usepackage{minted}
\usepackage{amssymb}
\usepackage{enumerate}
\usepackage{relsize}
\usepackage{tocloft} %Those cute dots on TOC

\renewcommand{\cftsecleader}{\cftdotfill{\cftdotsep}}

\newmintedfile[cppcodefile]{cpp}{linenos=true,frame=leftline,framesep=2mm,tabsize=4}
\newminted{cpp}{linenos=true,frame=leftline,framesep=2mm}
\pagestyle{fancy}

\fancyhead{}
\fancyfoot{}
\renewcommand{\sectionmark}[1]{\markright{#1}}
\fancyhead[R]{\rightmark}
\fancyhead[L]{Competitive Programming Notebook : \textbf{Data Structures}}
\renewcommand{\footrulewidth}{0.4pt}
\fancyfoot[R] {\thepage}


\title{Competitive Programming Notebook : Data Structures}
\author{Santiago Baldrich}
\date{v1.0}
\begin{document}
\tableofcontents
\newpage

%:Start

%:Trees
\section{Trees}
%:Fenwick

\subsection{Fenwick trees}
Provides efficient methods for calculation and manipulation of prefix sums. Excels over \textbf{Segment Trees} on ease of coding and memory usage. Yet there are problems for which the Segment tree is a better choice.

\cppcodefile{../code/cpp/ds/trees/fenwick.cpp}

\subsubsection{Field test: Interval Product }
\textbf{Source}:\textit{Latin America Regional Contest - 2012}\\
Up to $10^5$ numbers are given. One must update any value or answer if the product between all numbers in any interval $[i,j]$ is positive, negative or zero.

\cppcodefile{../code/cpp/ds/trees/fenwick_ex.cpp}\label{fenwick:ex}

%:Tries

\subsection{Tries}
A Trie (\textit{its name is an infix of re\textbf{trie}val}) is a powerful structure that can find or insert strings in $O(L)$ time where $L$ is the length of the string.\\
 I know, it's a shitty intro but Tries are a really useful data structure that allows answering important questions about the nature of a dictionary. They're usually the answer to questions as \textit{how to implement autocompletion on a web browser? } or \textit{given a misspelled word, what's the most probable word that could have been intended?}
 
\cppcodefile{../code/cpp/ds/trees/trie.cpp}

\subsubsection{Field test: Cellphone Typing }\label{trie:ex}
\textbf{Source}:\textit{Latin America Regional Contest - 2012}\\
A set of words is given as the dictionary of a cellphone auto-complete algorithm. Return the average number of keystrokes a user must use to write the words in the dictionary.\\

\cppcodefile{../code/cpp/ds/trees/trie_ex.cpp}

\subsection{Segment Trees}
A Segment Tree is a heap-like data structure that allows update/query operations in logarithmical time. They can help solve a wider range of problems than Fenwick Trees but use more memory (about 4 times more) and take a little more time to be coded.

\cppcodefile{../code/cpp/ds/trees/segment.cpp}

\subsubsection{Field test: Interval Product }
\textbf{Source}:\textit{Latin America Regional Contest - 2012}\\
\textit{The same problem solved in \ref{fenwick:ex} with Fenwick Trees but using Segment Trees.}\\
Up to $10^5$ numbers are given. One must update any value or answer if the product between all numbers in any interval $[i,j]$ is positive, negative or zero.

\cppcodefile{../code/cpp/ds/trees/segment_ex.cpp}

\subsection{Huffman Trees}
Huffman coding is a compression algorithm that takes into account the frequency or probability of appearance of each symbol to create a space efficient compression scheme.
\subsubsection{Field test: Entropy }
\textbf{Source}:\textit{Live Archive - 2088}\\
Determine the compression ratio between an ASCII encoded string (8bits per character) and an optimal compression (via huffman coding).
\cppcodefile{../code/cpp/ds/trees/huffman_ex.cpp}


\end{document}  













